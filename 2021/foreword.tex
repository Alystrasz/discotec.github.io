% \documentclass{article}
\documentclass[10pt]{article}

\title{Foreword}
\date{}

\pagestyle{empty}


\begin{document}

\maketitle

The 16th International Federated Conference on Distributed Computing Techniques (DisCoTec) took place in June 14–18, 2021. 
%
It was organised by the Department of Computer Science at the University of Malta, but was held online due to the abnormal circumstances worldwide affecting physical travel.
%
The DisCoTec series is one of the major events sponsored by the International Federation for Information Processing (IFIP), the European Association for Programming Languages and Systems (EAPLS) and the Microservices Community. It comprises three conferences:
\begin{itemize}
  \item \emph{COORDINATION}, the IFIP WG 6.1 23rd International Conference on Coordination Models and Languages;
  \item \emph{DAIS}, the IFIP WG 6.1 21st International Conference on Distributed Applications and Interoperable Systems;
  \item \emph{FORTE}, the IFIP WG 6.1 41st International Conference on Formal Techniques for Distributed Objects, Components and Systems.
\end{itemize}

Together, these conferences cover a broad spectrum of distributed computing subjects, ranging from theoretical foundations and formal description techniques to systems research issues. 
%
As is customary, the event also included several plenary sessions in addition to the individual sessions of each conference, that gathered attendants from the three conferences.  
%
These included joint invited speaker sessions and a joint session for the best papers from the respective three conferences. 
%
Associated with the federated event, four satellite events took place:
\begin{itemize}
  \item \emph{DisCoTec Tool}, a tutorial session promoting mature tools in the field of Distributed Computing;
  \item  \emph{ICE}, the 14th International Workshop on Interaction and Concurrency Experience; 
  \item \emph{FOCODILE}, the 2nd International Workshop on Foundations of Consensus and Distributed Ledgers;
  \item \emph{REMV}, the 1st Robotics, Electronics, and Machine Vision Workshop. 
\end{itemize}


I would like to thank the Program Committee chairs of the different events for their help and cooperation during the preparation of the conference, and the Steering Committee and Advisory Boards of DisCoTec and their conferences for their guidance and support. 
%
The organization of DisCoTec 2021 was only possible thanks to the dedicated work of the Organizing Committee, including Caroline Caruana and Jasmine Xuereb (publicity chairs),  Duncan Paul Attard and Christian Bartolo Burlo  (workshop chairs), Lucienne Bugeja (logistics and finances), as well as all the students and colleagues who volunteered their time to help. 
%
I would also like to thank the invited speakers for their excellent talks.
%
Finally, I would like to thank IFIP WG 6.1, EAPLS and the Microservices Community for sponsoring this event, Springer’s Lecture Notes in Computer Science team for their support and sponsorship, EasyChair for providing the reviewing framework, and the University of Malta for providing the support and infrastructure to host the event.

\vspace{0.6cm}
\noindent 
Adrian Francalanza

\noindent
General Chair

\noindent
June, 2021

\end{document}